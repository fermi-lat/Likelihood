\documentclass[preprint]{aastex}  
%\documentclass[iop]{emulateapj}
%\usepackage{booktabs,caption,fixltx2e}
\usepackage{natbib}
\bibliographystyle{aa}

\usepackage{graphicx,color,rotating}
\usepackage{footnote,lineno}
\usepackage{ulem} 
\usepackage{xspace}
%\linenumbers
%%%%%%%%%%%%%%%%%%%%%%%%%%%%%%%%%%%%%%%%
%\usepackage{txfonts}
\usepackage{graphicx,amssymb,amsmath,amsfonts,times,hyperref}
%\usepackage{rotating} 

%\linenumbers

\def \aap  {A\&A}

\newcommand{\fermipy}{\texttt{Fermipy}\xspace}
\newcommand{\newtext}[1]{{\color{blue}{#1}}}


%\usepackage{epsfig,epstopdf}
%%%%%%%%%%%%%%%%%%%%%%%%%%%%%%%%%%%%%%%%
%
\begin{document}
%
\title{Notes on Binned Likelihood Calculation in the Fermi-LAT Science Tools.}  

\author{ 
Some folks
%E.~Charles\altaffilmark{1}, 
%J.~Chiang\altaffilmark{1}, 
}
\altaffiltext{11}{W. W. Hansen Experimental Physics Laboratory, Kavli Institute for Particle Astrophysics and Cosmology, Department of Physics and SLAC National Accelerator Laboratory, Stanford University, Stanford, CA 94305, USA}



\begin{abstract}
  This is a collection of notes and equations about the binned log likelihood calculation in the Fermi-LAT Science Tools.
\end{abstract}

%\pacs{}
\maketitle

%%%%%%%%%%%%
\section{Introduction}
%%%%%%%%%%%%

The negative log-likelihood is:
\begin{equation}
  - \ln \mathcal{L} = \sum_i^{\rm pix} \sum_k^{\rm energy}  N_{ik} \ln M_{ik} - M_{ik},
\end{equation}
where $N_{ik}$ and $M_{ik}$ are respectively the number of observed and model counts 
for pixel $i$ and energy bin $k$.  The sum runs over all pixels and energy bins in the analysis.

In the case of a weighted analysis this becomes:
\begin{equation}
  - W \ln \mathcal{L} = \sum_i^{\rm pix} \sum_k^{\rm energy}  W_{ik} \cdot ( N_{ik} \ln M_{ik} - M_{ik} )
\end{equation}
where $W_{ik}$ is the weight assigned to pixel and energy bin $ik$. 

There are a number of complications in the way that this is calculated.
\begin{enumerate}
\item For speed of evaluation we calculate the two terms, i.e, the $N_{ik} \ln M_{ik}$ and the $M_{ik}$ terms, 
  seperately, as we can take certain short cuts in the two different cases.
\item Some quantities are integral quantities, defined over the energy bins, and some
  are differential, defined at the bin edges.  We have to take care to treat the two
  cases properly. 
\item We need to retain the ability to deal with energy dispersion.
\item We need need to treat both the weighted and unweighted cases.
\item We precompute and cache many parts of the calculation.
\end{enumerate}


\section{The summation over $N \ln M$.}

For this term we note that only bins with observed counts contribute to the sum.
Therefore, in general we will indentify the ``filled'' pixels once and then
only sum over those. 

We can also merge the $W_{ik} N_{ik}$ into a ``weighted counts map'' with $\tilde{N_{ik}} = W_{ik} N_{ik}$.
We do this because there are a number of reasons to carry around the weighed counts map, 
e.g., for debugging and diagnostic purposes:
\begin{equation}
  \sum_i^{\rm pix} \sum_k^{\rm energy} \tilde{N_{ik}} \ln M_{ik},
\end{equation}
where in practice we will only sum over the ``filled'' pixels, where $\tilde{N_{ik}} \ne 0$.


\section{Computing the model counts: $M_i$.}

To compute the model counts we sum over the model counts, $M_{ijk}$ of the sources in the model, indexed by $j$:
\begin{equation}
  M_{ik} =  \sum_j^{\rm src} M_{ijk}.
\end{equation}
We also precompute the ``spatial'' part of the model counts $m_{ijk}$ , i.e., the spatial model convolved with the 
point-spread function (PSF) and the effective area $A_{\rm eff}$.  Those computations are discussed elsewhere.

We then use log-log quadrature to integrate the product of the spatial and spectral parts of the model over 
each energy bin.  This gives us:
\begin{equation}
  M_{ijk} = (m_{ijk} S_{jk} E_k +  m_{ijk+1} S_{jk+1} E_{k+1}) \cdot ln(E_{k+1}/E_{k}) / 2,
\end{equation}
where $m_{ijk}$ is the precomputed value of the ``source map'' and $S_{jk}$ is the spectral model for source $j$ at energy bin edge $k$.

Since logarithms are expensive, the term $r_k = ln(E_{k+1}/E_{k})$ is precomputed and cached for each $k$, giving us:
\begin{equation}
  M_{ijk} = (m_{ijk} S_{jk} E_k +  m_{ijk+1} S_{jk+1} E_{k+1}) r_k / 2.
\end{equation}

A key point is that we define $y_{ijk} = m_{ijk} S_{jk}$ actually combine them between sources.  This speeds up the evaluation 
a bit, (and makes our live a lot more complicated).   Before we cached the $r_k$ it probably sped things up a lot.  
These $y_{ijk}$ have gotten the name ``model weights'' which is unfourtunate because it is easily confused with the likelihood weights.
\begin{equation}
  M_{ijk} = (y_{ijk} E_k +  y_{ijk+1} E_{k+1}) r_k / 2.
\end{equation}

We can sum the model weights over the sources:
\begin{equation}
  Y_{ik} = \sum_j^{\rm src} y_{ijk}.
\end{equation}

We can compute the model counts using the summed model weights:
\begin{equation}
  M_{ik} = (Y_{ik} E_k +  Y_{ik+1} E_{k+1}) r_k / 2.
\end{equation}

To speed up the calculation we seperately compute and cache the model weights for the fixed sources:
\begin{equation}
  Y_{ik}^{\rm fixed} = \sum_j^{\rm fixed} y_{ijk}.
\end{equation}
So that the total model weights is:
\begin{equation}
  Y_{ik} = Y_{ik}^{\rm fixed} + \sum_j^{\rm free} y_{ijk},
\end{equation}


\section{Applying the energy dispersion.}

To apply the energy dispersion we first precompute the detector response matrix (DRM, $D_{kl}$) 
that gives the chances for a photon in true energy bin $l$ to be observed in energy bin $k$.
We then compute the unconvolved energy spectrum for a source,
given the current model by summing the model over all the pixels for that energy layer:
\begin{equation}
  U_{jl} = \sum_i^{\rm pix} M_{ijl}.
\end{equation}
And use that to compute the convolved spectrum:
\begin{equation}
  C_{jk} = \sum_l^{\rm energies} D_{kl} U_{jl}.
\end{equation}
We then take the ratio of the convolved to unconvolved spectra in each energy bin (i.e., 
for $k$=$l$) as a correction factor for that energy for that source:
\begin{equation}
  \xi_{jk} = C_{jk} U_{jk},
\end{equation}
and use this to adjust the model energy spectrum from true energy to measure energy:
\begin{equation}
  M_{ijk} = \xi_{jk} M_{ijk}^{\rm true}.
\end{equation}
So that the equation for the model becomes:
\begin{equation}
  M_{ijk} = \xi_{jk} \cdot (m_{ijk} S_{jk} E_k +  m_{ijk+1} S_{jk+1} E_{k+1}) r_k / 2.
\end{equation}
The mismatch between then number of energy bins and energy bin edges (i.e., the difference in index
on $\xi$ and the other terms) makes things a bit more complicated, now we need
two weights for each bin:
\begin{eqnarray}
  y_{ijk}^{-} & = & \xi_{jk} m_{ijk} S_{jk} \nonumber \\
  y_{ijk}^{+} & = & \xi_{jk} m_{ijk+1} S_{jk+1}.
\end{eqnarray}
And then we can write the model as:
\begin{equation}
  M_{ijk} = ( y_{ijk}^{-} E_k + y_{ijk}^{+} E_{k+1} ) r_k / 2.
\end{equation}
And then the total model weights and fixed model weights also come in two parts:
\begin{eqnarray}
  Y_{ik}^{-} & = & Y_{ik}^{-,\rm fixed} + \sum_j^{\rm free} y_{ijk}^{-} \nonumber \\
  Y_{ik}^{+} & = & Y_{ik}^{+,\rm fixed} + \sum_j^{\rm free} y_{ijk}^{+}.
\end{eqnarray}
So that the total model value in a given bin is then:
\begin{equation}
  M_{ik} = ( Y_{ik}^{-} E_k + Y_{ik}^{+} E_{k+1} ) r_k / 2.
\end{equation}


\section{The summation over $M$.}

For this term we all the bins with observed counts contribute to the sum,
but we not that we can use an something similar to the model weights to speed things
up.  Specifically, the sum of the source maps over all the pixels in a energy layer
doesn't change as we are fitting the sources.  This sum is called ``Npred'' in the
code, which is unfortunate, as it isn't really a predicted number of counts, but rather
something that you can used to get the predicted number of counts.  Specifically 
we can define the sums:
\begin{equation}
  P_{jk} = \sum_i m_{ijk},
\end{equation}

Then the unconvolved and convolved specturm for a free source $j$ become:
\begin{eqnarray}
  U_{jk} = ( P_{jk} S_{jk} E_k + P_{jk+1} S_{jk+1} E_{k+1} ) r_k / 2  \nonumber \\
  C_{jk} = \xi_{jk} ( P_{jk} S_{jk} E_k + P_{jk+1} S_{jk+1} E_{k+1} ) r_k / 2.
\end{eqnarray}

For fixed sources we can calcule the sum of the npreds, including the spectral tern, 
and the average values of $\xi$ for the fixed sources
\begin{eqnarray}
  P_{k}^{\rm fixed} & = & \sum_j^{\rm fixed} S_{jk} P_{jk} \nonumber \\
  P_{k}^{-, \rm fixed} & = & \sum_j^{\rm fixed} \xi_{jk} S_{jk} P_{jk}  \nonumber \\
  P_{k}^{+, \rm fixed} & = & \sum_j^{\rm fixed} \xi_{jk} S_{jk+1} P_{jk+1}.
\end{eqnarray}

Then the unconvolved and convolved specturm for the sum of all the fixed sources are
\begin{eqnarray}
  U_{k}^{\rm fixed} = ( P_{k}^{\rm fixed} E_k + P_{k+1}^{\rm fixed}  E_{k+1} )  r_k / 2  \nonumber \\
  C_{k}^{\rm fixed} = ( P_{k}^{-, \rm fixed} E_k + P_{k}^{+, \rm fixed} E_{k+1} )  r_k / 2.
\end{eqnarray}

And the total unconvolved and convolved specturm are:
\begin{eqnarray}
  U_{k} = U_{k}^{\rm fixed} + \sum_j^{\rm free} U_{jk} \nonumber \\
  C_{k} = C_{k}^{\rm fixed} + \sum_j^{\rm free} C_{jk} 
\end{eqnarray}


\section{Applying Weights to the summation over $M$.}

In this case we have to account for the weights when we sum over the pixels
\begin{equation}
  WP_{jk} = \sum_i W_{ik} m_{ijk}.
\end{equation}

We want to be able to switch between the weighted and unweighted case easily, 
so we compute the ratio of the weighted to unweighted sums:
\begin{eqnarray}
  R_{jk}^{-} & = & \sum_i W_{ik} m_{ijk} / P_{jk} \nonumber \\
  R_{jk}^{+} & = & \sum_i W_{ik} m_{ijk+1} / P_{jk+1} \nonumber \\
\end{eqnarray}

Then the weighted unconvolved and convolved specturm for a free source $j$ become:
\begin{eqnarray}
  WU_{jk} = ( R_{jk}^{-} P_{jk} S_{jk} E_k + R_{jk}^{+} P_{jk+1} S_{jk+1} E_{k+1} )  r_k / 2  \nonumber \\
  WC_{jk} = ( R_{jk}^{-} \xi_{jk} P_{jk} S_{jk} E_k + R_{jk}^{+} \xi_{jk} P_{jk+1} S_{jk+1} E_{k+1} )  r_k / 2.
\end{eqnarray}

For fixed sources we want we also need the weighted npreds for the true and measured spectra:
\begin{eqnarray}
  WP_{k}^{-, \rm fixed} & = & \sum_j^{\rm fixed} R_{jk}^{-} P_{jk} S_{jk}  \nonumber \\
  WP_{k}^{+, \rm fixed} & = & \sum_j^{\rm fixed} R_{jk}^{+} P_{jk+1} S_{jk+1}  \nonumber \\
  WP_{k}^{-, \rm efixed} & = & \sum_j^{\rm fixed} \xi_{jk} R_{jk}^{-} P_{jk} S_{jk}  \nonumber \\
  WP_{k}^{+, \rm efixed} & = & \sum_j^{\rm fixed} \xi_{jk} R_{jk}^{+} P_{jk+1} S_{jk+1}.  
\end{eqnarray}

Then the weighted unconvolved and convolved specturm for the sum of all the fixed sources are
\begin{eqnarray}
  WU_{k}^{\rm fixed} = ( WP_{k}^{-, \rm fixed} E_k + WP_{k}^{+, \rm fixed} E_{k+1} )  r_k / 2  \nonumber \\
  WC_{k}^{\rm fixed} = ( WP_{k}^{-, \rm efixed} E_k + WP_{k}^{+, \rm efixed} E_{k+1} )  r_k / 2.
\end{eqnarray}

And the total unconvolved and convolved specturm are:
\begin{eqnarray}
  WU_{k} = WU_{k}^{\rm fixed} + \sum_j^{\rm free} WU_{jk} \nonumber \\
  WC_{k} = WC_{k}^{\rm fixed} + \sum_j^{\rm free} WC_{jk} 
\end{eqnarray}


%%%%%%%%%%%%
\section{Analytic derivatives}
%%%%%%%%%%%%

In general want the derivatives of the weighted log likelihood w.r.t., the 
free parameters:

\begin{equation}
  \frac{-\partial W \ln \mathcal{L}}{\partial x_{\alpha}} = \sum_i^{\rm pix} \sum_k^{\rm energy} (W_{ik} N_{ik} / M_{ik}) \frac{\partial M_{ik}}{\partial x_{\alpha}} - W_{ik} \frac{\partial M_{ik}}{\partial x_{\alpha}}.
\end{equation}
As before we can replace the observed counts with the weighed observed counts $\tilde{N_{ik}} = W_{ik} N_{ik}$ and split the 
evaluation into two sums, one over the filled pixels, the second over the derivatives of the total models counts 
for each of the free sources.


\section{The derivative $\frac{\partial M_{ik}}{\partial x_{\alpha}}$}

We need to evalute the derivate $\frac{\partial M_{ik}}{\partial x_{\alpha}}$.  We can ignore the fixed sources, as they won't contribute to the partial derivative.  Also, in general we are only varying spectral parameters, so only the spectral terms (the $S_{jk}$) are relevant.  (This neglects the effect of the change in the energy disperison corrections: $\xi_{jk}$, but that is a higher order effect.)  Finally, in the first term this is multiplied by $\tilde{N_{ik}}$ so we only need to sum over the filled pixels.  
\begin{equation}
  \frac{\partial M_{ijk}}{\partial x_{\alpha}} = \xi_{jk} \cdot (m_{ijk} \frac{\partial S_{jk}}{\partial alpha} E_k +  m_{ijk+1} \frac{\partial S_{jk+1}}{\partial alpha} E_{k+1}) r_k / 2.
\end{equation}
We pre-compute and cache the various $\delta_{jk\alpha} = \frac{\partial S_{jk}}{\partial x_{\alpha}}$ so we have:
\begin{equation}
  \frac{\partial M_{ijk}}{\partial x_{\alpha}} = \xi_{jk} \cdot (m_{ijk} \delta_{jk\alpha} E_k +  m_{ijk+1} \delta_{jk+1\alpha} E_{k+1}) r_k / 2.
\end{equation}
All of these terms except for the $\delta_{jk\alpha}$ were already computed in evaluating the likelihood.  We do need to make check if any of the 
parameter values have changed since the evaluation of the likelihood.  


\section{The derivative of  $\frac{\partial M}{\partial x_{\alpha}}$}

This is the derivative of the second term in the log likelihood.  Here we need to sum over all the pixels, but we only need to evaluate this for the 
free sources.  

The derivative of the weighted unconvolved and convolved specturm for a free source $j$:
\begin{eqnarray}
  WU_{jk} = ( R_{jk}^{-} P_{jk} \delta_{jk\alpha} E_k + R_{jk}^{+} P_{jk+1} \delta_{jk+1\alpha} E_{k+1} )  r_k / 2  \nonumber \\
  WC_{jk} = ( R_{jk}^{-} \xi_{jk} P_{jk} \delta_{jk\alpha} E_k + R_{jk}^{+} \xi_{jk} P_{jk} \delta_{jk+1\alpha} E_{k+1} )  r_k / 2.
\end{eqnarray}

As with the first term, all of these terms except for the $\delta_{jk\alpha}$ were already computed in evaluating the likelihood.   

%%%%%%%%%%%%%%%%%
\begin{acknowledgements}
  Thanks.
\end{acknowledgements}

%\bibliography{paper}

\appendix*

\section{Access to various quantities.}

These are the accesss functions used to retrieve various quatities.
\begin{itemize}
\item{ {\bf $N_{ik}$}: BinnedLikelihood::countsMap(). }
\item{ {\bf $\tilde{N_{ik}}$}: BinnedLikelihood::dataCache().weightedCounts(), which returns a pointer that may be null.}
\item{ {\bf $E_k$}: BinnedLikelihood::energies(). }
\item{ {\bf $r_{k}$}: BinnedLikelihood::dataCache(). }
\item{ {\bf $W_{ik}$}: BinnedLikelihood::weightMap(), which returns a pointer that may be null.}
\item{ {\bf $M_{ik}$}: BinnedLikelihood::computeModelMap(modelMap) fills modelMap with the $M_{ik}$ values.}
\item{ {\bf $M_{ijk}$}: BinnedLikelihood:computeModelMap(name, modelMap) fills modelMap with the $M_{ijk}$ values.}
\item{ {\bf $S_{jk}$}: BinnedLikelihood::sourceMap(name).cached\_specVals().}
\item{ {\bf $U_{jk}$}: BinnedLikelihood::sourceMap(name).cached\_drm\_cache()->true\_counts().}
\item{ {\bf $C_{jk}$}: BinnedLikelihood::sourceMap(name).cached\_drm\_cache()->meas\_counts().}
\item{ {\bf $\xi_{jk}$}: BinnedLikelihood::sourceMap(name).cached\_drm\_cache()->xi().}
\item{ {\bf $Y_{ik}^{\pm,\rm fixed}$}: BinnedLikelihood::fixedModelWts(); this is a vector that has the values for filled pixels only.}
\item{ {\bf $P_{jk}$}: BinnedLikelihood::sourceMap(name).cached\_npreds().}
\item{ {\bf $R_{jk}^{\pm}$}: BinnedLikelihood::sourceMap(name).cached\_npred\_weights().}
\item{ {\bf $P_{k}^{\rm fixed}$}: BinnedLikelihood::fixedNpreds().}
\item{ {\bf $P_{k}^{\pm, \rm fixed}$}: BinnedLikelihood::fixedNpred\_xis().}
\item{ {\bf $WP_{k}^{\pm, \rm fixed}$}: BinnedLikelihood::fixedNpred\_wts().}
\item{ {\bf $WP_{k}^{\pm, \rm efixed}$}: BinnedLikelihood::fixedNpred\_xiwts().}
\item{ {\bf $WU_{jk}$}: BinnedLikelihood::sourceMap(name).cached\_drm\_cache()->true\_counts\_wt().}
\item{ {\bf $WC_{jk}$}: BinnedLikelihood::sourceMap(name).cached\_drm\_cache()->meas\_counts\_wt().}
\item{ {\bf $\delta_{jk\alpha}$}: BinnedLikelihood::sourceMap(name).cached\_specDerivs(), returns the cached the spectral derivatives.}
\end{itemize}



\section{Functions and caching}

These are the functions used to calculate and cache various quantities.

\begin{itemize}
\item{ {\bf BinnedLikelihood::value()}: returns the negative log likelihood value, $-\ln \mathcal{L}$.}
\item{ {\bf BinnedLikelihood::getFreeDerivs(derivs)}: fills derivs with the derivatives of the 
    negative log likelihood, $\frac{\partial \-\ln \mathcal{L}}{\partial x_{\alpha}}$.}
\item{ {\bf BinnedLikelihood::computeModelMap\_internal()}: computes and caches the $M_{ik}$, $Y_{ik}^{\pm,\rm fixed}$, $P_{k}^{\rm fixed}$,
    $WP_{k}^{\rm fixed}$ and everything needed to compute those.}
\item{ {\bf BinnedLikelihood::buildFixedModelWts()}: computes and caches the $Y_{ik}^{\pm,\rm fixed}$, $P_{k}^{\pm, \rm fixed}$,
    $WP_{k}^{\rm fixed}$ and everything needed to compute those.}
\item{ {\bf SourceMap::setSpectralValues(energies)}: computes and caches the $S_{jk}$.}
\item{ {\bf SourceMap::setSpectralDerivs(energies, paramNames)}: computes and caches the $\delta_{jk\alpha}$.}
\item{ {\bf SourceMap::computeNpredArray()}: computes and caches the $U_{jk}$, $C_{jk}$, $\xi_{jk}$ $WU_{jk}$, and $WC_{jk}$.}
\item{ {\bf SourceMap::update\_drm\_cache(drm)}: computes and caches the $U_{jk}$, $C_{jk}$, $\xi_{jk}$ $WU_{jk}$, and $WC_{jk}$.}
\end{itemize}



\end{document}
 
