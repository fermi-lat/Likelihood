\documentclass[preprint]{aastex}  
%\documentclass[iop]{emulateapj}
%\usepackage{booktabs,caption,fixltx2e}
\usepackage{natbib}
\bibliographystyle{aa}

\usepackage{graphicx,color,rotating}
\usepackage{footnote,lineno}
\usepackage{ulem} 
\usepackage{xspace}
%\linenumbers
%%%%%%%%%%%%%%%%%%%%%%%%%%%%%%%%%%%%%%%%
%\usepackage{txfonts}
\usepackage{graphicx,amssymb,amsmath,amsfonts,times,hyperref}
%\usepackage{rotating} 

%\linenumbers

\def \aap  {A\&A}

\newcommand{\fermipy}{\texttt{Fermipy}\xspace}
\newcommand{\newtext}[1]{{\color{blue}{#1}}}


%\usepackage{epsfig,epstopdf}
%%%%%%%%%%%%%%%%%%%%%%%%%%%%%%%%%%%%%%%%
%
\begin{document}
%
\title{Notes on Binned Likelihood Calculation in the Fermi-LAT Science Tools.}  

\author{ 
Some folks
%E.~Charles\altaffilmark{1}, 
%J.~Chiang\altaffilmark{1}, 
}
\altaffiltext{11}{W. W. Hansen Experimental Physics Laboratory, Kavli Institute for Particle Astrophysics and Cosmology, Department of Physics and SLAC National Accelerator Laboratory, Stanford University, Stanford, CA 94305, USA}



\begin{abstract}
  This is a collection of notes and equations about the binned log likelihood calculation in the Fermi-LAT Science Tools.
\end{abstract}

%\pacs{}
\maketitle

%%%%%%%%%%%%
\section{Introduction}
%%%%%%%%%%%%

The negative log-likelihood is:

\begin{equation}
  - \ln \mathcal{L} = \sum_i^{\rm pix} \sum_k^{\rm energy}  N_{ik} \ln M_{ik} - M_{ik},
\end{equation}

\noindent where $N_{ik}$ and $M_{ik}$ are respectively the number of observed and model counts 
for pixel $i$ and energy bin $k$.  The sum runs over all pixels and energy bins in the analysis.

In the case of a weighted analysis this becomes:

\begin{equation}
  - \ln \tilde{\mathcal{L}} = \sum_i^{\rm pix} \sum_k^{\rm energy}  w_{ik} \cdot ( N_{ik} \ln M_{ik} - M_{ik} ),
\end{equation}

\noindent where $w_{ik}$ is the weight assigned to pixel and energy bin $ik$. 

We also want to calculate the derivatives of the likelihood with respect to the
free parameters ($x_{\alpha}$):

\begin{equation}
\frac{-\partial \ln \tilde{\mathcal{L}}}{\partial x_{\alpha}} = \sum_i^{\rm pix} \sum_k^{\rm energy} \frac{w_{ik} N_{ik}}{M_{ik}} \frac{\partial M_{ik}}{\partial x_{\alpha}} - w_{ik} \frac{\partial M_{ik}}{\partial x_{\alpha}}.
\end{equation}


There are a number of complications in the way that these quantities
are calculated.
\begin{enumerate}
\item The spatial binning can be either WCS or HEALPix based.
\item Some quantities are integral quantities, defined over the energy
  bins, and some are differential, defined at the bin edges.  We have
  to take care to treat the two cases properly.
\item We need need to treat both the weighted and unweighted cases.
\item We need to retain the ability to deal with energy dispersion.
\item For speed of evaluation we calculate the summation terms
  separately. This allows us to take certain short cuts in
  the two different cases.
\item For speed, we also precompute and cache many parts of the
  calculation.
\end{enumerate}

In this note we are going to ignore the difference between WCS and
HEALPix based maps.  For our purposes this just means that the pixel
index $i$ runs over the flattened map in the WCS case.

In principle we should be working with the energy bin edges: $E_k^-$,
$E_k^+$.  However, we always use connected energy bins, so that the
lower edge of one energy bin is the upper edge of the previous bin:
$E_k \equiv E_k^- = E_{k-1}^+$.


\section{Evaluation of the likelihood function}\label{sec:like}

Most of the work in evaluation the likelihood function comes in
computing the model counts (\S\ref{subsec:like_model_counts} and
\S\ref{subsec:like_model_edisp}).  Then, as stated above, we calculate
the two terms in the likelihood expression separately: the first is
the summation over $ w_{ik} N_{ik} \ln M_{ik}$ (\S\ref{subsec:like_sum1}) and
the second is the summation over $w_{ik} M_{ik}$ (\S\ref{subsec:like_sum2}).


\subsection{Computing the model counts: \texorpdfstring{$M$}{M}.}\label{subsec:like_model_counts}

To compute the model counts we sum over the model counts, $M_{ijk}$ of
the sources in the model, indexed by $j$:

\begin{equation}
  M_{ik} =  \sum_j^{\rm src} M_{ijk}.
\end{equation}

\noindent We also precompute the ``spatial'' part of the model counts
$m_{ijk}$ , i.e., the spatial model convolved with the point-spread
function (PSF) and the effective area $A_{\rm eff}$.  Those
computations are discussed elsewhere.

We then use log-log quadrature to integrate the product of the spatial
and spectral parts of the model over each energy bin.  This gives us:

\begin{equation}
  M_{ijk} = (m_{ijk} S_{jk} E_k +  m_{ijk+1} S_{jk+1} E_{k+1}) \cdot \frac{1}{2} \ln\frac{E_{k+1}}{E_{k}},
\end{equation}

\noindent where $m_{ijk}$ is the precomputed value of the ``source
map'' and $S_{jk}$ is the spectral model for source $j$ at energy bin
edge $k$.

Since logarithms are expensive, we precompute the term $r_{k} = \ln\frac{E_{k+1}}{E_{k}}$. 
Also, for each source we note the product $S_{jk} E_K \frac{r_{k}}{2}$ only depends
on the energy bin, so we can precompute and cache two ``spectral weights'' ($\tilde{S}_{jk}^{\pm}$) 
for each energy bin:

\begin{eqnarray}
  \tilde{S}_{jk}^{-} = S_{jk} E_k \frac{r_{k}}{2}, \nonumber \\
  \tilde{S}_{jk}^{+} = S_{jk+1} E_{k+1} \frac{r_{k}}{2}. 
\end{eqnarray}

\noindent Then we can rewrite the model counts as:

\begin{equation}
  M_{ijk} = m_{ijk} \tilde{S}_{jk}^{-}  + m_{ijk+1} \tilde{S}_{jk}^{+}. 
\end{equation}

\noindent We have reduced the number of operations (and the complexity of those operations) in the innermost loop
of the likelihood evalation.  

To speed up the calculation we separately compute and cache
the model counts for the all of the fixed sources:

\begin{equation}
  M_{ik}^{\rm fixed} = \sum_j^{\rm fixed} M_{ijk}.
\end{equation}

\noindent So that the total model weights are:

\begin{equation}
  M_{ik} = M_{ik}^{\rm fixed} + \sum_j^{\rm free} M_{ijk}.
\end{equation}

We can also save the fixed counts spectrum, $C_{j}^{\rm fixed}$:

\begin{equation}
  C_{k}^{\rm fixed} = \sum_i^{\rm pix} M_{ik}^{\rm fixed}.
\end{equation}




\subsection{Applying the energy dispersion.}\label{subsec:like_model_edisp}

To apply the energy dispersion we first precompute the detector
response matrix (DRM, $D_{kl}$) that gives the chances for a photon in
true energy bin $l$ to be observed in energy bin $k$.  

We can then apply the energy dispersion in one of two different ways, 
depending on the configuration parameters.


\subsection{Applying the energy dispersion with scaling}\label{subsec:like_model_edisp_0}

In this case we first compute the unconvolved energy spectrum for a source, given the current model
by summing the model over all the pixels for that energy layer:

\begin{equation}
  U_{jl} = \sum_i^{\rm pix} M_{ijl}.
\end{equation}

\noindent And use that to compute the convolved spectrum:

\begin{equation}
  C_{jk} = \sum_l^{\rm energies} D_{kl} U_{jl}.
\end{equation}

For each source we take the ratio of the convolved to unconvolved
spectra in each energy bin (i.e., for $k$=$l$) as a correction factor
for that energy for that source:

\begin{equation}
  \xi_{jk} = C_{jk} U_{jk}.
\end{equation}

\noindent We use this to adjust the model energy spectrum from true
energy to measured energy:

\begin{equation}
  M_{ijk} = \xi_{jk} M_{ijk}^{\rm true}.
\end{equation}

\noindent So that the equation for the model becomes:

\begin{equation}
  M_{ijk} = \xi_{jk} \cdot (m_{ijk} S_{jk} E_k +  m_{ijk+1} S_{jk+1} E_{k+1}) \frac{r_k}{2}.
\end{equation}

\noindent Then we can write the model in terms of the spectral weights as:

\begin{equation}
  M_{ijk} =  \xi_{jk} \cdot (m_{ijk} \tilde{S}_{jk}^{-}  + m_{ijk+1} \tilde{S}_{jk}^{+}).
\end{equation}

This method has the flaw that it does not properly account for the changing PSF as 
a function of energy.  It merely rescales the counts in the spectral domain using
the $\xi_{jk}$ parameters.


\subsection{Applying the energy dispersion with adjacent bins}\label{subsec:like_model_edisp_1}

To more correctly deal with energy dispersion in conjunction with the PSF, we should
actually apply the summation over $D_{kl}$ for every single pixel.  This will greatly 
slow down the execution time.  So as a compromise we only do the sum over the nearest
energy bins, which are most likely to contribute:

\begin{equation}
  M_{ijk} = \sum_l^{\rm k\pm n} D_{kl} M_{ijl}^{\rm true}.
\end{equation}

\noindent Where the sum over true energy bins runs from $k-n$ to $k+n$ where $n$ is user-configurable and
typically 1 or 2.

Expanding this in term of the spectral weights we obtain:

\begin{equation}
  M_{ijk} = \sum_l^{\rm k\pm n} D_{kl} (m_{ijl} \tilde{S}_{jl}^{-}  + m_{ijl+1} \tilde{S}_{jl}^{+}).
\end{equation}


\subsection{The summation over \texorpdfstring{$W N \ln M$}{WN ln M}.}\label{subsec:like_sum1}

For this term we note that only bins with observed counts contribute
to the sum.  Therefore, in general we will identify the ``filled''
pixels once and then only sum over those.

We can merge the $W_{ik} N_{ik}$ into a ``weighted counts map'' with
$\tilde{N_{ik}} = W_{ik} N_{ik}$.  We do this because there are a
number of reasons to carry around the weighed counts map, e.g., for
debugging and diagnostic purposes, and because it is more efficient to
use the weighted counts map in many computations.

\begin{equation}
  \sum_i^{\rm pix} \sum_k^{\rm energy} \tilde{N_{ik}} \ln M_{ik},
\end{equation}

\noindent where in practice we will only sum over the ``filled''
pixels, where $\tilde{N_{ik}} \ne 0$.  This also means that 
pixels with zero weight will be ignored in the summation over
filled pixels. 


\subsection{The summation over \texorpdfstring{$M$}{M}.}\label{subsec:like_sum2}

For this term we all the bins with observed counts contribute to the
sum, but we note that we can use something similar to the model
weights to speed things up.  Specifically, the sum of the source maps
over all the pixels in a energy layer does not change as we are fitting
the sources.  This sum is called ``Npred'' in the code, which is
unfortunate, as it is not really a predicted number of counts, but
rather something that you can used to get the predicted number of
counts.  Specifically we can define the sums:

\begin{equation}
  P_{jk} = \sum_i m_{ijk},
\end{equation}

\noindent Then the unconvolved and convolved spectra for a free
source $j$ become:

\begin{eqnarray}\label{eq:spectra_no_weights}
  U_{jk} = (P_{jk} S_{jk} E_k + P_{jk+1} S_{jk+1} E_{k+1})\frac{r_k}{2},  \nonumber \\
  C_{jk} = \xi_{jk} ( P_{jk} S_{jk} E_k + P_{jk+1} S_{jk+1} E_{k+1} ) \frac{r_k}{2}, \nonumber \\
  C_{jk} = \sum_l^{\rm k\pm n} D_{kl}  (P_{jl} S_{jl} E_l + P_{jl+1} S_{jl+1} E_{l+1})\frac{r_l}{2}.
\end{eqnarray}

\noindent Where we pick the second or third equation depending on how we are applying
the energy dispersion.

We can rewrite this in terms of the spectra weights as:

\begin{eqnarray}\label{eq:spectra_no_weights}
  U_{jk} = P_{jk} \tilde{S}_{jk}^{-} + P_{jk+1} \tilde{S}_{jk}^{+},  \nonumber \\
  C_{jk} = \xi_{jk} ( P_{jk} \tilde{S}_{jk}^{-} + P_{jk+1} \tilde{S}_{jk}^{+} ), \nonumber \\
  C_{jk} = \sum_l^{\rm k\pm n} D_{kl}  (P_{jl} \tilde{S}_{jl}^{-} + P_{jl+1} \tilde{S}_{jl}^{+} ).
\end{eqnarray}


\subsection{Applying weights to the summation over \texorpdfstring{$M$}{M}.}\label{subsec:like_sum2_weights}

In this case we have to account for the weights when we sum over the
pixels.  Also, we want to be able to switch between the weighted and
unweighted case easily, so we compute the weighted npreds on either
side of the energy bin:

\begin{eqnarray}
  \tilde{P}_{jk}^{-} & = & \sum_i w_{ik} m_{ijk} \nonumber \\
  \tilde{P}_{jk}^{+} & = & \sum_i w_{ik} m_{ijk+1}. 
\end{eqnarray}

\noindent Then the weighted unconvolved and convolved spectrum for a 
free source $j$ if we are using the rescaling method become:

\begin{eqnarray}
  \tilde{U}_{jk} = ( \tilde{P}_{jk}^{-} S_{jk} E_k + \tilde{P}_{jk}^{+} S_{jk+1} E_{k+1} )  \frac{r_k}{2},  \nonumber \\
  \tilde{C}_{jk} = \xi_{jk} ( \tilde{P}_{jk}^{-} S_{jk} E_k + \tilde{P}_{jk}^{+} S_{jk+1} E_{k+1} ) \frac{r_k}{2}.
\end{eqnarray}

\noindent The only differences with respect to Eq.~\ref{eq:spectra_no_weights} 
are the $\tilde{P}_{jk}$ terms.  Of course we can rewrite this in terms of the spectral weights: 

\begin{eqnarray}
  \tilde{U}_{jk} = \tilde{P}_{jk}^{-}  \tilde{S}_{jk}^{-}  + \tilde{P}_{jk}^{+} \tilde{S}_{jk}^{+},  \nonumber \\
  \tilde{C}_{jk} = \xi_{jk} ( \tilde{P}_{jk}^{-}  \tilde{S}_{jk}^{-}  + \tilde{P}_{jk}^{+} \tilde{S}_{jk}^{+} ).
\end{eqnarray}

If we are using the adjacent bins energy dispersion treatment this becomes quite
a bit more complicated.  This is because we must account for the energy dispersion when
computing the weighted npreds.  

\begin{eqnarray}
  \tilde{P}_{jkl}^{-} & = & \sum_i w_{ik} \sum_l^{\rm k\pm n} D_{kl} m_{ijl} \nonumber \\
  \tilde{P}_{jkl}^{+} & = & \sum_i w_{ik} \sum_l^{\rm k\pm n} D_{kl} m_{ijl+1}. 
\end{eqnarray}

\noindent Where, once again the sum over true energy bins ($l$) runs from $k-n$ to $k+n$ where $n$ is user-configurable and
typically 1 or 2.  Then the weighted, convolved spectra become:

\begin{equation}
  \tilde{C}_{jk} = \sum_l^{\rm k\pm n} \tilde{P}_{jkl}^{-}  \tilde{S}_{jl}^{-}  + \tilde{P}_{jkl}^{+} \tilde{S}_{jl}^{+}.
\end{equation}


\subsection{Caching various fixed source spectra.}\label{subsec:like_sum2}

We note that we can save time by caching the
convolved and unconvolved and weighed and unweighted version of the fixed counts spectra:

\begin{eqnarray}
  U_{k}^{\rm fixed} = \sum_j^{\rm fixed} U_{jk},  \nonumber \\
  C_{k}^{\rm fixed} = \sum_j^{\rm fixed} C_{jk},  \nonumber \\
  \tilde{U}_{k}^{\rm fixed} = \sum_j^{\rm fixed} \tilde{U}_{jk},  \nonumber \\
  \tilde{C}_{k}^{\rm fixed} = \sum_j^{\rm fixed} \tilde{C}_{jk}. 
\end{eqnarray}


\section{Analytic derivatives}\label{sec:derivatives}

In general, we want the derivatives of the weighted log likelihood w.r.t.,
the free parameters:

\begin{equation}\label{eq:derivatives}
  \frac{-\partial \ln \tilde{\mathcal{L}}}{\partial x_{\alpha}} = \sum_i^{\rm pix} \sum_k^{\rm energy} \frac{w_{ik} N_{ik}}{M_{ik}} \frac{\partial M_{ik}}{\partial x_{\alpha}} - w_{ik} \frac{\partial M_{ik}}{\partial x_{\alpha}}.
\end{equation}

\noindent As before, we can replace the observed counts with the
weighed observed counts $\tilde{N_{ik}} = w_{ik} N_{ik}$ and split the
evaluation into two sums, one over the filled pixels
(\S\ref{subsec:derivatives_part1}), the second over the derivatives of
the total models counts for each of the free sources
(\S\ref{subsec:derivatives_part2}).


\subsection{The derivatives \texorpdfstring{$\frac{\partial M_{ijk}}{\partial x_{\alpha}}$}{dM/dx}}\label{subsec:derivatives_part1}

The first term of Eq.~\ref{eq:derivatives} is:

\begin{equation}
  \sum_i^{\rm pix} \sum_k^{\rm energy} \frac{w_{ik} N_{ik}}{M_{ik}} \frac{\partial M_{ik}}{\partial x_{\alpha}} = 
  \sum_i^{\rm pix} \sum_k^{\rm energy} \frac{\tilde{N}_{ik}}{M_{ik}} \sum_j^{\rm src} \frac{\partial M_{ijk}}{\partial x_{\alpha}}.
\end{equation}

We write it like this because we have already computed $M_{ij}$ so we
do not need to re-do the sums over sources there.  However, we need to
evaluate the derivatives $\frac{\partial M_{ijk}}{\partial x_{\alpha}}$.
We can ignore the fixed sources, as they won't contribute to the
partial derivative.  Also, in general we are only varying spectral
parameters, so only the spectral terms (the $S_{jk}$) are relevant.
(This neglects the effect of the change in the energy dispersion
corrections: $\xi_{jk}$, but that is a higher order effect.)  Finally,
in the first term this is multiplied by $\tilde{N_{ik}}$ so we only
need to sum over the filled pixels.  If we are using the rescaling
method of handling the energy dispersion this yields:

\begin{equation}
  \frac{\partial M_{ijk}}{\partial x_{\alpha}} = \xi_{jk} \cdot (m_{ijk} \frac{\partial S_{jk}}{\partial x_{\alpha}} E_k +  m_{ijk+1} \frac{\partial S_{jk+1}}{\partial x_{\alpha}} E_{k+1}) \frac{r_k}{2}.
\end{equation}

\noindent We pre-compute and cache the various $\delta_{jk\alpha} = \frac{\partial S_{jk}}{\partial x_{\alpha}}$ so we have:

\begin{equation}
  \frac{\partial M_{ijk}}{\partial x_{\alpha}} = \xi_{jk} \cdot (m_{ijk} \delta_{jk\alpha} E_k +  m_{ijk+1} \delta_{jk+1\alpha} E_{k+1}) \frac{r_k}{2}.
\end{equation}

\noindent All of these terms except for the $\delta_{jk\alpha}$ were
already computed in evaluating the likelihood.  We do need to make
check if any of the parameter values have changed since the evaluation
of the likelihood, and possibly updated the cached $M_{ik}$ in the 
prefactor.   We can further speed up the evaluation by pre-computing the terms
that depend only on the energy bin:

\begin{eqnarray}
  \tilde{\delta}_{jk\alpha}^{-} = \delta_{jk\alpha} E_k \frac{r_k}{2}, \nonumber \\
  \tilde{\delta}_{jk\alpha}^{+} = \delta_{jk+1\alpha} E_{k+1} \frac{r_k}{2}.
\end{eqnarray}

\noindent So that we are evaluating:

\begin{equation}
  \frac{\partial M_{ijk}}{\partial x_{\alpha}} = \xi_{jk} \cdot (m_{ijk} \tilde{\delta}_{jk\alpha}^{-} +  m_{ijk+1} \tilde{\delta}_{jk\alpha}^{+})
\end{equation}

If, on the other hand, we are using the adjacent bins method of handling the energy dispersion we have 
to sum over the nearby true energy bins:

\begin{equation}
  \frac{\partial M_{ijk}}{\partial x_{\alpha}} =  \sum_l^{\rm k\pm n} D_{kl} (m_{ijl} \frac{\partial S_{jl}}{\partial x_{\alpha}} E_l +  m_{ijl+1} \frac{\partial S_{jl+1}}{\partial x_{\alpha}} E_{l+1}) \frac{r_l}{2}.
\end{equation}

\noindent Or, in terms of the precomputed quantities:

\begin{equation}
  \frac{\partial M_{ijk}}{\partial x_{\alpha}} =  \sum_l^{\rm k\pm n} D_{kl} (m_{ijl} \tilde{\delta}_{jl\alpha}^{-} +  m_{ijl+1} \tilde{\delta}_{jl\alpha}^{+} ).
\end{equation}



\subsection{The derivative \texorpdfstring{$\frac{\partial M}{\partial x_{\alpha}}$}{dM/dx}}\label{subsec:derivatives_part2}

The second term of Eq.~\ref{eq:derivatives} is:

\begin{equation}
  \sum_i^{\rm pix} \sum_k^{\rm energy} w_{ik}\frac{\partial M_{ik}}{\partial x_{\alpha}} = 
  \sum_k^{\rm energy} \sum_j^{\rm src} \sum_i^{\rm pix} w_{ik} \frac{\partial M_{ijk}}{\partial x_{\alpha}},
\end{equation}

\noindent where we need to sum over all the pixels, but we only need
to evaluate this for the free sources.  Since we are summing over all
the pixels, we can express the derivatives $\frac{\partial
  M_{ijk}}{\partial x_{\alpha}}$ in terms of $\tilde{U}_{jk}$ and
$\tilde{C}_{jk}$ which we computed in the likelihood evaluation:

\begin{eqnarray}
  \sum_i^{\rm pix} \frac{\partial M_{ijk}^{\rm true}}{\partial x_{\alpha}} = \frac{\partial\tilde{U}_{jk}}{\partial x_{\alpha}}, \nonumber \\
  \sum_i^{\rm pix} \frac{\partial M_{ijk}^{\rm meas}}{\partial x_{\alpha}} = \frac{\partial\tilde{C}_{jk}}{\partial x_{\alpha}}.
\end{eqnarray}


The derivative of the weighted unconvolved and convolved spectrum for
a free source $j$:
\begin{eqnarray}
  \frac{\partial\tilde{U}_{jk}}{\partial x_{\alpha}} & = & ( P_{jk}^{-} \delta_{jk\alpha} E_k + P_{jk+1}^{+} \delta_{jk+1\alpha} E_{k+1} )  \frac{r_k}{2}, \nonumber \\
  \frac{\partial\tilde{C}_{jk}}{\partial x_{\alpha}} & = & \xi_{jk} ( \tilde{P}_{jk}^{-}  \delta_{jk\alpha} E_k + \tilde{P}_{jk}^{+} \delta_{jk+1\alpha} E_{k+1} ) \frac{r_k}{2}, \nonumber \\
  \frac{\partial\tilde{C}_{jk}}{\partial x_{\alpha}} & = & \sum_l^{\rm k\pm n} D_{kl} ( \tilde{P}_{jkl}^{-}  \delta_{jl\alpha} E_l + \tilde{P}_{jkl}^{+} \delta_{jl+1\alpha} E_{l+1} ) \frac{r_l}{2}.
\end{eqnarray}

\noindent Where the second and third form are for the rescaling and adjacent bins versions of the energy dispersion, respectively.  
We can re-write those in terms of the spectral derivative weights, $\tilde{\delta}_{jk\alpha}^{\pm}$ as:

\begin{eqnarray}
  \frac{\partial\tilde{U}_{jk}}{\partial x_{\alpha}} & = & P_{jk}^{-} \tilde{\delta}_{jk\alpha}^{-} + P_{jk}^{+} \tilde{\delta}_{jk\alpha}^{+},  \nonumber \\
  \frac{\partial\tilde{C}_{jk}}{\partial x_{\alpha}} & = & \xi_{jk} ( P_{jk}^{-} \tilde{\delta}_{jk\alpha}^{-} + P_{jk}^{+} \tilde{\delta}_{jk\alpha}^{+} ), \nonumber \\
  \frac{\partial\tilde{C}_{jk}}{\partial x_{\alpha}} & = & \sum_l^{\rm k\pm n} D_{kl} ( P_{jl}^{-} \tilde{\delta}_{jl\alpha}^{-} + P_{jl}^{+} \tilde{\delta}_{jl\alpha}^{+} )
\end{eqnarray}

As with the first term, all of these terms except for the
$\delta_{jk\alpha}$ were already computed in evaluating the
likelihood.

\begin{acknowledgements}
  Thanks.
\end{acknowledgements}

%\bibliography{paper}

\appendix

\section{Access to various quantities.}\label{app:access}

These are the access functions used to retrieve various quantities.
These all assume that the user has an object of class {\tt
  BinnedLikelihood} stored in the variable {\tt blike}.  Some of them
refer to quantities associated with a particular source with {\tt
  name}.

\begin{itemize}
\item{ {\bf $N_{ik}$}: {\tt blike.countsMap()}. Returns a pointer
    to an object of class {\tt CountsMapBase}.}
\item{ {\bf $\tilde{N_{ik}}$}: {\tt
      blike.dataCache().weightedCounts()}.  Returns a pointer to
    an object of class {\tt CountsMapBase} that may be null if no
    weights have been supplied.}
\item{ {\bf $E_k$}: {\tt blike.energies()}. Returns a vector of
    doubles, whose length is equal to the number of energy bin edges.}
\item{ {\bf $r_{k}$}: {\tt blike.dataCache().log\_energy\_ratios()}.
    Returns a vector of doubles, whose length is equal to the
    number of energy bins.}
\item{ {\bf $w_{ik}$}: {\tt blike.weightMap()}.  Returns a
    pointer to an object of class {\tt WeightMap} that may be null if
    no weights have been supplied.}
\item{ {\bf $M_{ik}$}: {\tt blike.computeModelMap(modelMap)}.  Fills
    {\tt modelMap} with the $M_{ik}$ values and returns void.}
\item{ {\bf $M_{ijk}$}: {\tt blike.computeModelMap(name,
      modelMap)}.  Fills {\tt modelMap} with the $M_{ijk}$ values
    for the source in question and returns void.}
\item{ {\bf $M_{ik}^{\rm fixed}$}: {\tt blike.fixedModelCounts()}.
    Returns a vector of doubles, for only the filled pixels.}
\item{ {\bf $U_{k}^{\rm fixed}$}: {\tt blike.fixedModelSpectrum()}.  Returns
    a vector of doubles, whose length is equal to the number of energy
    bins.}
\item{ {\bf $\tilde{U}_{k}^{\rm fixed}$}: {\tt blike.fixedModelSpectrum\_wt()}.  Returns
    a vector of doubles, whose length is equal to the number of energy
    bins.}
\item{ {\bf $C_{k}^{\rm fixed}$}: {\tt blike.fixedModelSpectrum\_edisp()}.  Returns
    a vector of doubles, whose length is equal to the number of energy
    bins.}
\item{ {\bf $\tilde{C}_{k}^{\rm fixed}$}: {\tt blike.fixedModelSpectrum\_edisp\_wt()}.  Returns
    a vector of doubles, whose length is equal to the number of energy
    bins.}
\item{ {\bf $m_{ijk}$}: {\tt blike.sourceMap(name).cached\_model()}.
    Returns a vector of floats, whose length is equal to the
    number of energy bin edges times the number of pixels.}
\item{ {\bf $S_{jk}$}: {\tt blike.sourceMap(name).cached\_specVals()}.
    Returns a vector of doubles, whose length is equal to the
    number of energy bin edges.}
\item{ {\bf $\tilde{S}_{jk}^{\pm}$}: {\tt blike.sourceMap(name).cached\_specWts()}.
    Returns a vector of pairs of doubles, whose length is equal to the
    number of energy bins.}
\item{ {\bf $P_{jk}$}: {\tt blike.sourceMap(name).cached\_npreds()}.
    Returns a vector of doubles, whose length is equal to the
    number of energy bin edges.}
\item{ {\bf $P_{jkl}^{\pm}$}: {\tt blike.sourceMap(name).cached\_weighted\_npreds()}.
    Returns a vector of vector of pairs of doubles, whose length is equal to the
    number of energy bin edges.  Each sub-vector has length equal to the number
    of bins being used in the energy dispersion calculation.}
\item{ {\bf $U_{jk}$}: {\tt
      blike.sourceMap(name).cached\_drm\_cache()->true\_counts()}.
    Returns a vector of doubles, whose length is equal to the
    number of energy bins.}
\item{ {\bf $C_{jk}$}: {\tt
      blike.sourceMap(name).cached\_drm\_cache()->meas\_counts()}.
    Returns a vector of doubles, whose length is equal to the
    number of energy bins.}
\item{ {\bf $\xi_{jk}$}: {\tt
      blike.sourceMap(name).cached\_drm\_cache()->xi()}.  Returns
    a vector of doubles, whose length is equal to the number of energy
    bins.}
\item{ {\bf $\tilde{U}_{jk}$}: {\tt
      blike.sourceMap(name).cached\_drm\_cache()->true\_counts\_wt()}.
    Returns a vector of doubles, whose length is equal to the number
    of energy bins.}
\item{ {\bf $\tilde{C}_{jk}$}:
    {\tt blike.sourceMap(name).cached\_drm\_cache()->meas\_counts\_wt().}.
    Returns a vector of doubles, whose length is equal to the
    number of energy bins.}
\item{ {\bf $\delta_{jk\alpha}$}: {\tt
      blike.sourceMap(name).cached\_specDerivs()}.  Returns a
    vector of vector of doubles, whose dimensions are the number of
    free parameters (for that source) and the number of energy bin edges.}
\end{itemize}



\section{Functions and caching}\label{app:functions}

These are the higher level functions used to calculate and cache various quantities.
Some of the functions assume that the user has an object of class {\tt
  SourceMap} stored in the variable {\tt srcMap}.

\begin{itemize}
\item{ {\bf {\tt blike.value()}}: returns the negative log
    likelihood value, $-\ln \mathcal{L}$.}
\item{ {\bf {\tt blike.getFreeDerivs(derivs)}}: fills derivs
    with the derivatives of the negative log likelihood,
    $\frac{\partial \-\ln \tilde{\mathcal{L}}}{\partial x_{\alpha}}$.}
\item{ {\bf {\tt blike.computeModelMap\_internal()}}: computes
    and caches the $M_{ik}$, $U_{k}^{\rm fixed}$, $C_{k}^{\rm fixed}$, $\tilde{U}_{k}^{\rm fixed}$, $\tilde{C}_{k}^{\rm fixed}$ 
    and everything needed to compute those.}
\item{ {\bf {\tt blike.buildFixedModelWts()}}: computes and
    caches the $M_{ik}^{\rm fixed}, U_{k}^{\rm fixed}$, $C_{k}^{\rm fixed}$, $\tilde{U}_{k}^{\rm fixed}$, $\tilde{C}_{k}^{\rm fixed}$ and everything needed to compute those.}
\item{ {\bf {\tt srcMap.setSpectralValues(energies)}}: computes and
    caches the $S_{jk}$ and $\tilde{S}_{jk}^{\pm}$.}
\item{ {\bf {\tt srcMap.setSpectralDerivs(energies, paramNames)}}:
    computes and caches the $\delta_{jk\alpha}$.}
\item{ {\bf {\tt srcMap.computeNpredArray()}}: computes and caches the
    $U_{jk}$, $C_{jk}$, $\xi_{jk}$ $\tilde{U}_{jk}$, and $\tilde{C}_{jk}$.}
\item{ {\bf {\tt srcMap.update\_drm\_cache(drm)}}: computes and caches
    the $U_{jk}$, $C_{jk}$, $\xi_{jk}$ $\tilde{U}_{jk}$, and $\tilde{C}_{jk}$.}
\end{itemize}

These are the lower level functions used to calculate specific quantities.  These
are generally static functions in the {\tt FitUtils} namespace.

\begin{itemize}
\item{ {\bf {\tt FitUtils::get\_edisp\_range(const BinnedCountsCache\& dataCache,\\ \hspace{1in} 
        int edisp\_val, size\_t k, size\_t\& kmin, size\_t\& kmax)}}: finds the
    range of energy bins to sum over to apply the energy dispersion for a given bin $k$, and 
    fills kmin and kmax with the ends of that range.}
\item{ {\bf {\tt FitUtils::get\_edisp\_constants(SourceMap\& srcMap, \\ \hspace{1in} 
        const BinnedCountsCache\& dataCache,\\ \hspace{1in} 
        int edisp\_val, size\_t k, size\_t\& kmin, size\_t\& kmax,\\ \hspace{1in} 
        std::vector$\langle$double$\rangle$\& edisp\_col)}}: 
    depending on the energy dispersion implementation, fills the edisp\_col variable with the 
    values of either $D_{kl}$ or $\xi_{jk}$.  Note that the first value in edisp\_col correspond
    to the value for kmin.}
\item{ {\bf {\tt FitUtils::get\_spectral\_weights(const std::vector$\langle$double$\rangle$\& spec,\\ \hspace{1in} 
        const BinnedCountsCache\& dataCache,\\ \hspace{1in} 
        std::vector$\langle$std::pair$\langle$double, double$\rangle\rangle$\&spec\_wts)}}: combines
    the spectral values (spec, $s_{jk}$) with the energy binning (dataCache, $E_{k}$ and $r_{k}$) and computes
    the spectral weights ($\tilde{S}_{jk}^{\pm}$) and fills the spec\_wts vector of doubles
    with those weights.}
\item{ {\bf {\tt FitUtils::model\_counts\_contribution(SourceMap\& srcMap,\\ \hspace{1in}
        std::vector$\langle$std::pair$\langle$double, double$\rangle\rangle$\& spec\_wts, \\ \hspace{1in}
        const double\& xi, size\_t npix, size\_t kref, size\_t ipix)}}: finds 
    and returns the model counts contribution $m_{ijk}$ for a specific pixel and energy bin.  This implementation
    is used when energy dispersion is turned off, or when the rescaling method is used for 
    energy dispersion.}
\item{ {\bf {\tt FitUtils::model\_counts\_edisp(SourceMap\& srcMap,\\ \hspace{1in}
        std::vector$\langle$std::pair$\langle$double, double$\rangle\rangle$\& spec\_wts, \\ \hspace{1in}
        std::vector$\langle$double$\rangle$\& edisp\_col, \\ \hspace{1in}
        size\_t ipix, size\_t npix, size\_t kmin, size\_t kmax)}}: finds 
    and returns $m_{ijk}$ for a specific pixel and energy bin.  This implementation
    is used when energy dispersion adjacent bins method is used for energy dispersion.}
\item{ {\bf {\tt FitUtils::npred\_contribution(std::vector$\langle$double$\rangle$\& npred\_vals, \\ \hspace{1in}
        std::pair$\langle$double, double$\rangle$\& npred\_weights,  \\ \hspace{1in}
        std::vector$\langle$std::pair$\langle$double, double$\rangle\rangle$\& spec\_wts, \\ \hspace{1in} 
        const double\& xi, size\_t kref, double\& counts, double\& counts\_wt)}}: finds 
    both the weighted and unweighted counts for a specific energy bin and fill the count and counts\_wt variables with them. 
    This implementation is used when energy dispersion is turned off, or when the rescaling method is used for 
    energy dispersion.}
\item{ {\bf {\tt FitUtils::npred\_edisp(std::vector$\langle$double$\rangle$\& npred\_vals, \\ \hspace{1in}
        std::vector$\langle$std::pair$\langle$double, double$\rangle\rangle$\& npred\_weights, \\ \hspace{1in}
        std::vector$\langle$std::pair$\langle$double, double$\rangle\rangle$\& spec\_wts, \\ \hspace{1in}
        std::vector$\langle$double$\rangle$\& edisp\_col,\\ \hspace{1in}
        size\_t kmin, size\_t kmax, double\& counts, double\& counts\_wt)}}: finds 
    both the weighted and unweighted counts for a specific energy bin and fill the count and counts\_wt variables with them. 
    This implementation is used when energy dispersion adjacent bins method is used for energy dispersion..}
\item{ {\bf {\tt FitUtils::addSourceCounts(std::vector$\langle$double$\rangle$\& modelCounts,  \\ \hspace{1in}
        SouceMap\& srcMap, \\ \hspace{1in}
        const BinnedCountsCache\& dataCache, \\ \hspace{1in}
        int edisp\_val, bool subtract)}}: adds (or subtracts if subtract is true)
    the model counts for a particular source to the modelCounts vector.}
\item{ {\bf {\tt FitUtils::addFixedNpreds(std::vector$\langle$double$\rangle$\& fixed\_counts\_spec,   \\ \hspace{1in}
        std::vector$\langle$double$\rangle$\& fixed\_counts\_spec\_wt, \\ \hspace{1in}
        std::vector$\langle$double$\rangle$\& fixed\_counts\_spec\_edisp, \\ \hspace{1in}
        std::vector$\langle$double$\rangle$\& fixed\_counts\_spec\_edisp\_wt, \\ \hspace{1in}
        SouceMap\& srcMap, \\ \hspace{1in}
        const BinnedCountsCache\& dataCache,  \\ \hspace{1in}
        int edisp\_val, bool subtract)}}: adds (or subtracts if subtract is true)
    the various convolved and weighted versions of the npreds from a fixed source on the relevant vectors.}
\item{ {\bf {\tt FitUtils::addFreeDerivs(std::vector$\langle$Kahan\_Accumulator$\rangle$\& posDerivs,  \\ \hspace{1in}
        std::vector$\langle$Kahan\_Accumulator$\rangle$\& negDerivs,  \\ \hspace{1in}
        long freeIndex, SourceMap\& srcMap, \\ \hspace{1in}
        const std::vector$\langle$double$\rangle$\& data\_over\_model,  \\ \hspace{1in}
        const BinnedCountsCache\& dataCache,  \\ \hspace{1in}
        int edisp\_val, size\_t kmin, size\_t kmax)}}:
    adds the contributions to the derivatives of the log-likelihood for a single source.}  
\item{ {\bf {\tt FitUtils::updateModelMap(std::vector$\langle$float$\rangle$\& modelMap, \\ \hspace{1in}
        SourceMap\& srcMap, \\ \hspace{1in}
        const BinnedCountsCache\& dataCache, \\ \hspace{1in}
        bool use\_mask, int edisp\_val)}}: computes the model counts
    for a given source and fills the modelMap vector with them.  If use\_mask is true it will only fill the un-masked pixels.}
\end{itemize}


\end{document}
 

% LocalWords:  WCS HEALPix unweighted precompute convolved PSF precomputed DRM
% LocalWords:  unconvolved ijk jk ik efixed Eq weightedCounts modelMap specVals
% LocalWords:  computeModelMap drm npreds npred wts xiwts specDerivs derivs WN
% LocalWords:  paramNames BinnedLikelihood blike countsMap dataCache weightMap
% LocalWords:  sourceMap fixedModelWts fixedNpreds fixedNpred getFreeDerivs ln
% LocalWords:  buildFixedModelWts prefactor CountsMapBase WeightMap SourceMap
% LocalWords:  srcMap setSpectralValues setSpectralDerivs computeNpredArray jl
% LocalWords:  rescales rescaling jkl ijl dM dx fixedModelCounts edisp kmin un
% LocalWords:  fixedModelSpectrum kmax npix kref ipix vals modelCounts
% LocalWords:  posDerivs negDerivs freeIndex
