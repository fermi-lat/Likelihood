\documentclass[preprint]{aastex}  
%\documentclass[iop]{emulateapj}
%\usepackage{booktabs,caption,fixltx2e}
\usepackage{natbib}
\bibliographystyle{aa}

\usepackage{graphicx,color,rotating}
\usepackage{footnote,lineno}
\usepackage{ulem} 
\usepackage{xspace}
%\linenumbers
%%%%%%%%%%%%%%%%%%%%%%%%%%%%%%%%%%%%%%%%
%\usepackage{txfonts}
\usepackage{graphicx,amssymb,amsmath,amsfonts,times,hyperref}
%\usepackage{rotating} 

%\linenumbers

\def \aap  {A\&A}

\newcommand{\fermipy}{\texttt{Fermipy}\xspace}
\newcommand{\newtext}[1]{{\color{blue}{#1}}}


%\usepackage{epsfig,epstopdf}
%%%%%%%%%%%%%%%%%%%%%%%%%%%%%%%%%%%%%%%%
%
\begin{document}
%
\title{Notes on Binned Likelihood Calculation in the Fermi-LAT Science Tools.}  

\author{ 
Some folks
%E.~Charles\altaffilmark{1}, 
%J.~Chiang\altaffilmark{1}, 
}
\altaffiltext{11}{W. W. Hansen Experimental Physics Laboratory, Kavli Institute for Particle Astrophysics and Cosmology, Department of Physics and SLAC National Accelerator Laboratory, Stanford University, Stanford, CA 94305, USA}



\begin{abstract}
  This is a collection of notes and equations about the likelihood weighting implmentation.
\end{abstract}

%\pacs{}
\maketitle

%%%%%%%%%%%%
\section{Introduction}
%%%%%%%%%%%%

This implmentation and equations are based on Jean Ballet's work.


\section{The effective background: $B$}

The first step in compute the weights is to derive the ``effective
background'' for every pixel and energy bin.  This is essentially the
contribution of the background to the analysis of a point source at a
particular energy bin.

We derive the effective background staring some representation of the
counts in the region of interest (ROI).  This can be either binned
data, or a model of the ROI.  Following notation we used elsewhere
let's call this $M_{ik}$, where the $i$ index runs over the pixels in
the model, and the $k$ index runs over the energy bins.  The energy
bin egdes are at $E_k^-, E_k^+$ (typcially $E_k^+ = E_{k+1}^-$).  The
geometric energy bin centers are $E_k = (E_k^+ E_k^-)$.  The energy
bin widths are $\delta E_k = E_k^+ - E_k^-$, and the pixel sizes are
$\delta \Omega_i$.

In words, we want define the effective background $B_{ik}$ by
convolving $M_{ik}$ with the point-spread function at the each energy (PSF, $P_k$) and then
sum the result over all energies greater than or equal to a particular
energy.

\begin{equation}
B_{ik} = \sum_{j}^{j \ge k} \frac{M_{ij} \bigotimes P_{j}}{P_{j}^{\rm max}}, 
\end{equation}

\noindent where $P_{j}^{\rm max}$ is the maximum value of the PSF at energy $j$.

The convolution routines in the ScienceTools work on objects
of type {\tt ProjMap} (actually, the sub-classes {\tt HealpixProjMap}
and {\tt WcsMap2}), which are differential quantities (i.e., they are
intensities, defined at specific energy / directions, rather that
begin integrated across a range of energies and over a pixel.  In
practical terms, this just means that we have to convert the model
counts from the class {\tt CountsMapBase} to an object of the class
{\tt ProjMap}, we do this just by dividing the bin contents by the
energy bin widths and pixels sizes.

\begin{equation}
I_{ik} = \frac{M_{ik}}{\delta \Omega_i \delta E_k}.
\end{equation}

\noindent what the actually get back from the PSF convolution routine
is the normalized convolution:

\begin{equation}
\tilde{I}_{ik} = I_{ik} \bigotimes P_{k}.
\end{equation}

To get the effective background, we have to convert that quantity back
to counts and sum of all the energy bins greater than or equal to a particular
energy.

\begin{equation}
B_{ik} = \sum_{j}^{j \ge k} \frac{\tilde{I}_{ij} \delta E_k}{P_{j}^{\rm max}}.
\end{equation}

This quantity has units of counts, and is essentially a counts map.
We store it as such.  It can be produced by the standalone app {\tt
  gteffbkg} or using the {\tt pyLikelihood} interface.  The resulting
file will look almost identical to a binned counts map file, including
the {\tt EBOUNDS} and {\tt GTI} hdus, and the DSS keywords copied from
the input file.  The only difference will the addition of keywords to 
the primary header of the output file:

\begin{enumerate}
\item{{\tt MAPTYPE} will be set to ``BKG_EFF''.}
\item{{\tt INPUTMAP} which will give the name of the input binned counts map
file.}
\end{enumerate}

\section{The weighted sum over components: $\alpha$}

In order to properly deal with multiple analysis components we
need to compute the weighted sum over the components.  This 
quantity depend on the level of systematic error we are aiming from ($\epsilon$),
on the individual $B_{ikm}$ for each compoment (indexed by $m$), and on the
maximum $B_{ikm}$ for the various components for each pixel and energy bin: $\hat{B}_{ik}$.

We define this weighted sum as:

\begin{equation}
\alpha_{ik} = \frac{1 + \epsilon^2 \hat{B}_{ik}}{1 + \epsilon^2 \hat{B}_{ik} \sum_{m} (\frac{\hat{B}_{ik}}{B_{ikm}})^2 }.
\end{equation}

\noindent In the case that we are using a single component, then all of the $\alpha_{ik} \equiv 1$.

This quantity is dimensionless, but has the same binning as the input
counts map.  It can be produced by the standalone app {\tt gtalphabkg}
or using the {\tt pyLikelihood} interface.  This will store almost
exactly the same information as a {\tt CountsMap} or {\tt
  CountsMapHealpix}; the only differnce being the additional keywords:
\begin{enumerate}
\item{{\tt MAPTYPE} will be set to ``ALPHA_BKG''.}
\item{{\tt EPSILON}: giving the value of $\epsilon$ used in the computation;}
\item{{\tt BKGMAPXX}: will list the input effective background maps.}    
\end{enumerate}



\section{The likelihood weights: $w$}

Given the effective background maps and the $\alpha_{ik}$, the likelihood weights for a particular
component are defined as:

\begin{equation}
w_{ikm} = \frac{\alpha_{ik}}{1 + \epsilon^2 B_{ikm}}.
\end{equation}

This quantity is dimensionless, but has the same binning as the input counts map.   
It can be produced by the standalone app {\tt gtwtsmap} or using the {\tt pyLikelihood} interface.

This will store almost exactly the same information as a {\tt
  CountsMap} or {\tt CountsMapHealpix}; the only differnce being the
additional keywords:
\begin{enumerate}
\item{{\tt MAPTYPE} will be set to ``WEIGHT_MAP''.}
\item{{\tt EPSILON}: giving the value of $\epsilon$ used in the computation;}
\item{{\tt BKGMAP}: will give the input effective background map.}    
\item{{\tt ALPHAMAP}: the file containing the input $\alpha$ map (if used).}
\end{enumerate}


\section{Using the likelihood weights.}

The idea is that using the likelihood weights is almost transparent.  If a likelihood
weight are specified when the {\tt BinnedLikelihood} object is being created, those weights
will be used, otherwise no weights will be used.

For backward compatibility reasons the likelihood weights can be specified in number of ways.

\begin{enumerate}
\item{By passing an object of type {\tt WeightMap} into the constructor of {\tt BinnedLikelihood}.
    If the binning of the input {\tt WeightMap} does not match the binning of the {\tt CountsMap}
    used to construct the {\tt BinnedLikelihood} it will be resampled, taking the values
    from the bin and pixel centers of the {\tt CountsMap} and assigning the values 
    of the corresponding bin and pixel in the input {\tt WeightMap}.}
\item{By passing an object of type {\tt ProjMap} into the constructor of BinnedLikelihood.
    If the binning of the input {\tt ProjMap} does not match the binning of the {\tt CountsMap}
    used to construct the {\tt BinnedLikelihood} it will be resampled, taking the values
    from the bin and pixel centers of the {\tt CountsMap} and assigning the values 
    by averaging the nearest pixel values from the adjacent energy layers.  
    (This is because a {\tt ProjMap} is a differntial map, defined at specific energies.)}
\item{By passing a file name into the constructor of {\tt pyLikelihood.BinnedAnalysis}.
    This point to any file that contains either a {\tt WeightMap} or a {\tt ProjMap}, and the
    file will be handled according to the rules above.}
\item{By specifying a file name for the hidder {\tt wmap} parameter of {\tt gtlike} or {\tt gtscrmaps}.
    This point to any file that contains either a {\tt WeightMap} or a {\tt ProjMap}, and the
    file will be handled according to the rules above.}
\end{enumerate}

\end{document}
 
