\documentclass[preprint]{aastex}  
%\documentclass[iop]{emulateapj}
%\usepackage{booktabs,caption,fixltx2e}
\usepackage{natbib}
\bibliographystyle{aa}

\usepackage{graphicx,color,rotating}
\usepackage{footnote,lineno}
\usepackage{ulem} 
\usepackage{xspace}
%\linenumbers
%%%%%%%%%%%%%%%%%%%%%%%%%%%%%%%%%%%%%%%%
%\usepackage{txfonts}
\usepackage{graphicx,amssymb,amsmath,amsfonts,times,hyperref}
%\usepackage{rotating} 

%\linenumbers

\def \aap  {A\&A}

\newcommand{\fermipy}{\texttt{Fermipy}\xspace}
\newcommand{\FIXME}[1]{{\color{red}{#1}}}


%\usepackage{epsfig,epstopdf}
%%%%%%%%%%%%%%%%%%%%%%%%%%%%%%%%%%%%%%%%
%
\begin{document}
%
\title{Notes on the implementation of likelihood weighting in the ScienceTools.}  

\author{ 
Some folks
%E.~Charles\altaffilmark{1}, 
%J.~Ballet\altaffilmark{2}, 
}
\altaffiltext{1}{W. W. Hansen Experimental Physics Laboratory, Kavli Institute for Particle Astrophysics and Cosmology, Department of Physics and SLAC National Accelerator Laboratory, Stanford University, Stanford, CA 94305, USA}



\begin{abstract}
  This is a collection of notes and equations about the likelihood weighting implementation.
\end{abstract}

%\pacs{}
\maketitle

%%%%%%%%%%%%
\section{Introduction}
%%%%%%%%%%%%

This implementation and equations are based on Jean Ballet's work.\footnote{Extensive notes can be found at: \url{https://confluence.slac.stanford.edu/x/ZphdCw}}


\section{The effective background: $B$}

The first step in computing the weights is to derive the ``effective
background'' for every pixel and energy bin.  This is essentially the
contribution of the background to the analysis of a point source at a
particular energy bin.

We derive the effective background starting from some representation
of the counts in the region of interest (ROI).  This can be either
binned data, or a model of the ROI.  Following notation we used
elsewhere\footnote{Specifically, in the write up of the binned
  likelihood implementation.} let's call this $M_{ik}$, where the $i$
index runs over the pixels in the model, and the $k$ index runs over
the energy bins.  The energy bin edges are at $E_k^-, E_k^+$
(typically $E_k^+ = E_{k+1}^-$).  The geometric energy bin centers are
$E_k = \sqrt{E_k^+ E_k^-}$.  The energy bin widths are $\delta E_k = E_k^+
- E_k^-$, and the pixel sizes are $\delta \Omega_i$.

In words, we want to define the effective background $B_{ik}$ by
convolving $M_{ik}$ with the point-spread function (PSF) at each
energy $P_k$ and then sum the result over all energies greater than or
equal to a particular energy.

\begin{equation}\label{eq:bkg_eff_defintion}
B_{ik} = \sum_{j \ge k} \frac{M_{ij} \bigotimes P_{j}}{P_{j}^{\rm max}}, 
\end{equation}

\noindent where $P_{j}^{\rm max}$ is the maximum value of the PSF at energy $j$.
Another way of expressing this is that it is the PSF-weighted integral over
the background map.  

The convolution routines in the ScienceTools work on objects of type
{\tt ProjMap} (actually, the sub-classes {\tt HealpixProjMap} and {\tt
  WcsMap2}), which are differential quantities (i.e., they are
intensities, defined at specific energy / directions, rather than
begin integrated across a range of energies and over a pixel.  In
practical terms, this just means that we have to convert the model
counts from either {\tt CountsMap} or {\tt CountsMapHealpix} to a {\tt
  WcsMap2} or {\tt HealpixProjMap}, we do this just by dividing the
bin contents by the energy bin widths and pixels sizes.

\begin{equation}\label{eq:intensity}
I_{ik} = \frac{M_{ik}}{\delta \Omega_i \delta E_k}.
\end{equation}

\noindent What we actually get back from the PSF convolution routine
is the normalized convolution:

\begin{equation}\label{eq:convolved_intensity}
\tilde{I}_{ik} = I_{ik} \bigotimes P_{k}.
\end{equation}

To get the effective background, we have to convert that quantity back
to counts and sum of all the energy bins greater than or equal to a
particular energy.

\begin{equation}\label{eq:bkg_eff_computation}
B_{ik} = \sum_{j}^{j \ge k} \frac{\tilde{I}_{ij} \delta E_j}{P_{j}^{\rm max}}.
\end{equation}

\noindent Although the factors of $\delta E_k$ in Eq.~\ref{eq:intensity}
and $\delta E_j$ in Eq.~\ref{eq:bkg_eff_computation} cancel out, we have 
explicitly kept them in the notation, as they appear in the code because
of the way it is structured.

Th quantity $B_{ik}$ has units of counts, and is essentially a counts
map.  We store it as such.  It can be produced by the standalone
application {\tt gteffbkg} or using the {\tt pyLikelihood} interface.
The resulting file will look almost identical to a binned counts map
file, including the {\tt EBOUNDS} and {\tt GTI} HDUs, and the DSS
keywords are copied from the input file.  The only difference will be
the addition of keywords to the primary header of the output file:

\begin{enumerate}
\item{{\tt MAPTYPE} will be set to ``BKG\_EFF''.}
\item{{\tt INPUTMAP} which will give the name of the input binned counts map
file.}
\end{enumerate}

\section{The weighted sum over components: $\alpha$}

In order to properly deal with multiple analysis components we need to
compute the weighted sum over the components.  This quantity depends on
the level of systematic error we are considering ($\epsilon$), on the
individual $B_{ikm}$ for each component (indexed by $m$), and on the
minimum $B_{ikm}$ for the various components for each pixel and energy
bin: $\hat{B}_{ik}$.

We define this weighted sum as:

\begin{equation}
\alpha_{ik} = \frac{1 + \epsilon^2 \hat{B}_{ik}}{1 + \epsilon^2 \hat{B}_{ik} \sum_{m} (\frac{\hat{B}_{ik}}{B_{ikm}})^2 }.
\end{equation}

\noindent In the case that we are using a single component, then all
of the $\alpha_{ik} \equiv 1$.

This quantity is dimensionless, and has the same binning as the input
counts map.  It can be produced by the standalone app {\tt gtalphabkg}
or using the {\tt pyLikelihood} interface.  This will store almost
exactly the same information as a {\tt CountsMap} or {\tt
  CountsMapHealpix}. Since we are using several components, we remove 
the DSS keywords.  However we do add some keywords specifying the
provenance of the map:

\begin{enumerate}
\item{{\tt MAPTYPE} will be set to ``ALPHA\_BKG'';}
\item{{\tt EPSILON}: giving the value of $\epsilon$ used in the computation;}
\item{{\tt BKGMAPXX}: will list the input effective background maps.}    
\end{enumerate}



\section{The likelihood weights: $w$}

Given the effective background maps and the $\alpha_{ik}$, the
likelihood weights for a particular component are defined as:

\begin{equation}
w_{ikm} = \frac{\alpha_{ik}}{1 + \epsilon^2 B_{ikm}}.
\end{equation}

This quantity is dimensionless, but has the same binning as the input
counts map.  It can be produced by the standalone app {\tt gtwtsmap}
or using the {\tt pyLikelihood} interface.  However, for
``historical'' reasons, the {\tt BinnedLikelihood} object expects the
weights to be given as an object of the class {\tt ProjMap}.  In the
ScienceTools {\tt ProjMap} objects usually represent intensity maps,
but there is nothing enforcing this.  In practical terms the only real
difference for a weights file is that it is now defined at specific
energies rather than over energy ranges.  This means that we replace
the {\tt EBOUNDS} HDU with an {\tt ENERGIES} HDU.

Aside from that, writing the $w_{ik}$ will store almost exactly the
same information as a {\tt CountsMap} or {\tt CountsMapHealpix},
including the DSS keywords copied from the file with the $B_ik$; the
only differences being the additional keywords:

\begin{enumerate}
\item{{\tt MAPTYPE} will be set to ``WEIGHT\_MAP''.}
\item{{\tt EPSILON}: giving the value of $\epsilon$ used in the computation;}
\item{{\tt BKGMAP}: will give the input effective background map.}    
\item{{\tt ALPHAMAP}: the file containing the input $\alpha$ map (if used).}
\end{enumerate}


\section{Using the likelihood weights.}

The idea is that using the likelihood weights is almost transparent.
If a likelihood weights map is specified when the {\tt BinnedLikelihood}
object is being created, those weights will be used, otherwise no
weights will be used.

Depending on the interface used, the likelihood weights can be
specified in a number of ways.

\begin{enumerate}
\item{By passing an object of type {\tt ProjMap} (actually either a
    {\tt WcsMap2} or a {\tt HealpixProjMap}) into the constructor of
    {\tt BinnedLikelihood}.  The map will be re-sampled, taking the
    values from the bin and pixel centers of the {\tt CountsMapBase}
    and requesting the {\tt ProjMap} value for those.}
\item{By passing a file name into the constructor of {\tt
      pyLikelihood.BinnedAnalysis}.  This points to any file that
    contains a valid {\tt ProjMap}, and the file will be handled
    according to the rules above.}
\item{By specifying a file name for the hidden {\tt wmap} parameter of
    {\tt gtlike} or {\tt gtscrmaps}.  This points to any file that
    contains a valid {\tt ProjMap}, and the file will be handled
    according to the rules above.}
\end{enumerate}

\end{document}
 

% LocalWords:  ScienceTools ROI convolving PSF ProjMap HealpixProjMap WcsMap2
% LocalWords:  CountsMap CountsMapHealpix gteffbkg pyLikelihood EBOUNDS GTI DSS
% LocalWords:  HDUs MAPTYPE BKG INPUTMAP gtalphabkg BKGMAPXX gtwtsmap HDU wmap
% LocalWords:  BinnedLikelihood BKGMAP ALPHAMAP CountsMapBase BinnedAnalysis Eq
% LocalWords:  gtlike gtscrmaps
